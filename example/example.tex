% !TEX program=xelatex
% !TEX options=-synctex=1 -shell-escape -interaction=nonstopmode -file-line-error "%DOC%"
\documentclass[zihao=-4]{ctexart}
\usepackage[hmargin=2cm,vmargin=2.4cm]{geometry}
\usepackage[color,tikz]{texhigh}

\begin{document}

\title{高亮 \TeX 和 \LaTeX3 代码——使用\textsf{texhigh} 宏包}
\author{雾月\thanks{longaster@163.com}}
\maketitle

\textsf{texhigh} 宏包是专用来高亮 \TeX 文件的宏包。基于由 Rust 编写的命令行工具
texhigh\footnote{https://github.com/Sophanatprime/texhigh-rs},
处理 1.2M 左右(37000 余行)的 \texttt{expl3-code.tex} 只需不到 0.4s,
处理速度是 \textsf{minted} 宏包使用的 pygmentize 的 6 倍左右。
对于普通大小的 \TeX 代码,处理它们所需的时间相比于 \TeX 文件本身编译所需的时间,
已经可以忽略不记。

\textsf{texhigh} 主要是在 \LaTeX 中为 texhigh 命令行工具提供交互接口。这要求在编译 \TeX
文件时启用 \texttt{--shell-escape}。

\textsf{texhigh} 提供 
\texhighverb|\texhighverb、\texhighfile、\texhighinput|
这几个命令以及一个 \textsf{texhigh} 环境用于高亮 \TeX 代码。

\textsf{texhigh} 还有很强的可配置性。

为了实现处理 \TeX 源码与输出结果的分离,\textsf{texhigh} 使用“类型”和“类别”来区分不同的记号。
字符和控制序列是不同的“类型”,控制序列之间可以有不同的“类别”,例如是原语、\LaTeX3 函数等。
类型不可改变,而“类别”可以自由修改。

每个类型都有一些命令用于更改它们的“类别”的显示效果,如,对于一个控制序列,可以使用
\texhighverb|\THSetClassCS| 改变显示效果。可以为它们设置前景色、背景色,
甚至渐变色和底纹等等。实际上普通文字可以显示成什么效果,它们就可以做到同样的效果。

识别行内数学公式:\texhighverb!公式 $ \int_a^b x^2 dx = \frac{1}{3} x^3 |_a^b $!。

渐变:
\texhighverb[style=tikz.gradient, use-ctab=latex3, config-file=config.cfg]|\sys_get_shell:nnNTF|

底纹:
\makeatletter
\def\myshadetext#1#2{\texhigh@shadetext{#1}{\bfseries #2}}
\makeatother
{\LARGE
\texhighverb[use-ctab=latex3, this-cs=\myshadetext{fill stretch image=grass.png}]
  |\sys_get_shell:nnNTF|
}

中文命令识别:
\begin{texhigh}[output=\jobname.texhigh, use-ctab=cjk]
% 测试注释
\def\好好好{中文 Good}
\好好好\relax 
\end{texhigh}

\bigskip
\noindent\texhighverb|%%% 以下输出本文源码 %%%|
\texhighfile[style=tikz.gradient, use-ctab=cjkl3]{\jobname.tex}
\noindent\texhighverb|%%% 以上是本文源码 %%%|

\vspace{1cm}
\noindent\texhighverb|%%%---- File: texhigh.sty ----%%%|
\texhighfile[use-ctab=latex3code, config-file=config.cfg]{texhigh.sty}

\noindent\texhighverb|%%%---- File: prelude.ths ----%%%|
\texhighfile[use-ctab=latexcode, config-file=config.cfg]{prelude.ths}

\end{document}