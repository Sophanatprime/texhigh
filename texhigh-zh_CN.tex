\DocumentMetadata{lang=cn}
\documentclass[class=article,openany]{cusdoc}

\ExplSyntaxOn\makeatletter
\sys_if_shell_unrestricted:T
  {
    \usepackage{texhigh}
    \cs_if_exist:NTF \xeCJKsetup
      {
        \THSaveStyle{xamplestyle}{
          \SetKeys[texhigh/high]{config-file+=texhigh-doc.th.cfg,
            font=\xeCJKsetup{CJKecglue={\hskip 0pt plus 0.08\baselineskip}}\normalfont\ttfamily\small}
        }
      }
      {
        \THSaveStyle{xamplestyle}{
          \SetKeys[texhigh/high]{config-file+texhigh-doc.th.cfg, font=\normalfont\ttfamily\small}
        }
      }
    \THSaveStyle{cs-ref}
      {
        \THSetCS{?}{ \mbox {
          \tl_set:Nn \macro@escapepart {#1}
          \tl_set:Nn \macro@namepart {#2}
          \cus_doc_get_hyper_target:eeN \macro@namepart { function } \l_tmpa_tl
          \cs_if_exist:cT { r@ \l_tmpa_tl }
            {
              \hypersetup{hidelinks}
              \exp_args:NNo \hyperref [ \l_tmpa_tl ]
            }
              { \macro@escapepart \macro@namepart }
        } }
      }
    \RenewDocumentCommand{\xamplecodefile}{+m+m+m}
      {
        \IfBlankTF{#2}
          {\texhighfile[{style=plain,style=cs-ref,#1}]{#3}\par\vskip-5pt plus 1pt minus 1pt\relax}
          {\lstinputlisting[#1,#2]{#3}}
      }
  }

\NewDocumentCommand{\token}{ s o m}
  {
    \group_begin:
    \tl_if_novalue:nTF {#2}
      {
        \bool_if:nTF {#1}
          { \fboxsep2pt \fbox{ \ttfamily #3 } }
          { \texttt{ \c_backslash_str #3 } }
      }
      { \texttt {#3} \textsubscript{#2} }
    \group_end:
    \allowbreak
  }
\NewDocumentCommand\prolist{ O{,} O{\texttt} m }
  {
    \group_begin:
    \seq_set_from_clist:Nn \l_tmpa_seq {#3}
    \seq_pop_left:NNT \l_tmpa_seq \l_tmpa_tl
      {
        #2{ \l_tmpa_tl }
        \seq_map_inline:Nn \l_tmpa_seq { #1 #2{##1} }
      }
    \group_end:
  }

\NewDocumentEnvironment{thverbatim}{+!O{}}
  {\UseHook{env/texhigh/begin}\texhigh[gobble=auto,left-space=2cm,right-space=2cm,before=\smallskip,after=\nointerlineskip\smallskip,#1]}
  {\endtexhigh\@endparenv\UseHook{env/texhigh/end}}
\makeatother\ExplSyntaxOff

\hypersetup{pdfauthor={Longaster, \TeX high},
  pdftitle=\TeX high宏包手册,
  pdfcreator={\XeLaTeX} with hyperref and \CusLaTeX}

\makeatletter
\setuptitle[section]{format=\bfseries\Large,
  beforeskip=10pt plus 5pt minus 5pt,
  afterskip=10pt plus 5pt minus 5pt,
  fixskip, ensureskip,
}
\setuptitle[subsection]{format=\bfseries\large,
  beforeskip=7pt plus 5pt minus 5pt,
  afterskip=7pt plus 5pt minus 5pt,
  fixskip, ensureskip,
}
\setuplayout{paper=a4,
  marginparsep=10pt,marginparwidth=100pt,
  right=1.7cm,left=120pt,top=2.3cm,bottom=2.5cm,
  headsep=10pt,footskip=30pt,hfoffset=0pt,
}
\setpagestyle+{fancy}[totalempty]{\setfoot[c]{\thepage}}

\removegeneralname
\changes{v0.4.0}{2025/09/24}{添加文档。}
\begin{document}

\title{\TeX high —— 高亮 \TeX 源码}
\author{雾月\thanks{\texttt{longaster@163.com}}}
\date{\zhtoday\quad v0.4.0}
\maketitle

\enablecombinedlist
\def\contentsname{}
\templatetoc[
  section={space.left=1cm,space.hang=1cm,width.name=1cm},
  subsection={space.left=1.5cm,width.name=1cm,space.hang=1cm},
][columns=2,column-sep=6mm,outer-sep=0pt,ragged]

\bigskip

\pkg{texhigh} 是用于高亮 \TeX 代码的宏包,基于由 Rust 编写的命令行工具
\href{https://github.com/Sophanatprime/texhigh-rs}{texhigh},
\pkg{texhigh} 宏包是对该命令行工具的进一步封装。


\section{主要特性}

\pkg{texhigh} 的主要优势是速度快、可自定义程度高。

由于 texhigh 使用 Rust 编写,比起使用 Python 编写的 pygmentize 要快许多。
同样是处理 4 万行左右的 \texttt{expl3-code.tex} 文件,texhigh 只需 0.2s 左右,
而 pygmentize 则需 3.5s,有十几倍的差距。

\pkg{texhigh} 有两类配置。
一类是作用于 texhigh 命令行工具的(config,“配置”),它们用于控制如何解析要高亮的代码;
另一类是作用于 \TeX 输出的(style,“样式”),它们用于控制如何输出解析后的内容。

texhigh 仍然属于静态解析工具,对于 \TeX 这类上下文\emph{有关}的语言,
还无法完美地处理所有情形。尽管如此,\pkg{texhigh} 还是有许多可配置的地方。

texhigh 还有实验性质的计算文字布局(layout)的功能,通过它,可以输出 emoji 和颜文字等。


\section{主要命令}

\cs{texhighverb}、\cs{texhighfile} 这两个命令分别用于高亮 \TeX 代码和文件。
\cs{texhigh\-text} 也可用于高亮代码。\cs{texhighinput} 则用于输出已被 texhigh 处理好的内容。
\env{texhigh} 用于高亮该环境的内容。

\begin{function}{\texhighverb}
  \begin{syntax}
    \V\texhighverb \oarg{选项} \meta{token} \meta{tokens} \meta{token}
    \V\texhighverb \oarg{选项} \marg{balanced tokens}
  \end{syntax}
用法和 \tn{verb} 类似,用于高亮 \meta{tokens}。
\end{function}

\begin{function}{\texhightext}
  \begin{syntax}
    \V\texhightext \oarg{选项} \marg{balanced tokens}
  \end{syntax}
把 \meta{balanced tokens} 转为字符串,然后高亮它们。\cs{texhighverb} 内部也使用该命令。
\end{function}

我们可以用 \cs{texhightext} 来定义自己的高亮命令:
\begin{xample}
\NewDocumentCommand {\myverb} { v } {\texhightext[use-ctab=latex3]{#1}}

% 二者一样
\texhighverb[use-ctab=latex3] |\cs_set:Npn \__my_func:n #1 { \tl_to_str:n {#1} }|

\myverb|\cs_set:Npn \__my_func:n #1 { \tl_to_str:n {#1} }|
\stopxamplecode
\xampleprint
\end{xample}

\begin{function}{\texhighfile}
  \begin{syntax}
    \V\texhighfile \oarg{选项} \marg{文件}
  \end{syntax}
高亮 \meta{文件}。
\end{function}

\begin{function}{\texhighinput}
  \begin{syntax}
    \V\texhighinput \oarg{选项} \marg{文件}
  \end{syntax}
导入由 texhigh 命令行工具输出的文件。

例如,使用 \cs{texhighfile} \verb|[output=myout.thv]{somefile.tex}| 后,
可以用 \cs{texhigh\-input} \verb|{myout.thv}| 直接导入。
\end{function}

\begin{function}{\texhighdefshortverb}
  \begin{syntax}
    \V\texhighdefshortverb \oarg{选项} \meta{符号}
  \end{syntax}
把 \meta{符号} 局部地设置为“短的”高亮命令。如 \verb!| |! 那样。
\pkg{texhigh} 默认情况下不把任何符号设置为短的高亮命令。
\end{function}

\begin{xample}
\texhighdefshortverb\| \texhighdefshortverb!

|\def\nrelax{not \relax}| !\def\nrelax{\relax not}!
\stopxamplecode
\xampleprint
\end{xample}

\begin{function}[type=environment]{texhigh}
  \begin{syntax}
    \V*|\begin{texhigh}| \oarg{选项}
    ~~~~...
    \V*|\end{texhigh}|
  \end{syntax}
高亮环境内的内容。

如果要基于它来定义环境,则需使用命令的形式:
\begin{thverbatim}
\NewDocumentEnvironment{myverb}{O{}}
  {\texhigh[gobble=auto,#1]}
  {\endtexhigh}
\end{thverbatim}
并且不能使用 \texttt{b} 参数。

也可以将 \texttt{c} 参数与 \cs{texhighfile} 配合使用达到类似的效果,见\cref{eg:cmd-arg-c}。
\end{function}

\enlargethispage{10pt}
\begin{xample}
\ExplSyntaxOn
\tl_set:Nn \l__my_filename_tl { ./ \jobname.texhigh.verb }
\NewDocumentEnvironment{myverb}{c}
  { \UseHook{end/texhigh/begin} }
  {
    \iow_open:Nn \g_tmpa_iow { \l__my_filename_tl }
    \seq_set_split_keep_spaces:Nnn \l_tmpa_seq { \obeyedline } {#1}
    \seq_map_tokens:Nn \l_tmpa_seq { \iow_now:Nn \g_tmpa_iow }
    \iow_close:N \g_tmpa_iow
    \texhighfile [gobble=auto] { \l__my_filename_tl }
    \UseHook{end/texhigh/end}
  }
\ExplSyntaxOff

\begin{myverb}
  \def\foo{foo}
  \show\foo
\end{myverb}
\stopxamplecode
\xamplecode\label{eg:cmd-arg-c} \xampleline
\xampletext
\end{xample}


\section{选项}

\subsection{基本配置}

本小节的选项都是用于设置 texhigh 的配置的。

\begin{keyval}[path=high]{use-ctab}
  \begin{syntax}
    use-ctab = <catcode table> & document
  \end{syntax}
使用 \meta{catcode table} 来解析当前要高亮的代码。
字符的类别码(catcode)不同,解析得到的 token 也不同。

例如,使用 \tn{makeatletter} 会设置 \texttt{@} 的类别码为 11,这使得 \verb|\@firstofone|
被解析为 \token{@firstofone} 这个控制序列,而使用 \LaTeX 正文中默认的类别码时,则会解析为
控制序列 \token{@} 以及 \token[11]{f}\token[11]{i}\token[11]{r}\token[11]{s}\token[11]{t}\token[11]{o}\token[11]{f}\token[11]{o}\token[11]{n}\token[11]{e}。
解析得到的 token 不同,高亮结果也会有所差异。

\keyref*[high]{use-ctab} 就是用于设置解析时使用的类别码(更确切地说,是类别码表)。
可用的值有 \prolist{document,latex,latexcode,latex3,latex3code,cjk,cjkl3,cjkcode,cjkl3code} 等,还可以使用自定义的类别码表,见\cref{sec:ctab}。

\texttt{document} 和 \texttt{latex} 就是 \LaTeX 正文中默认使用的类别码表,
\texttt{latexcode} 在 \texttt{latex} 的基础上还设置了 \texttt{@} 的类别码为 11。
\texttt{latex3} 在 \texttt{latex} 的基础上使用 \cs{ExplSyntaxOn} 开启后的类别码。
\texttt{latex3code} 在 \texttt{latex3} 的基础上还设置了 \texttt{@} 的类别码为 11。
\texttt{cjk} 设置所有的 CJK 表意字符的类别码为 11。
\texttt{cjkl3} 在 \texttt{cjk} 的基础上还设置了 \texttt{@} 的类别码为 11。
\end{keyval}

\begin{keyval}[path=high]{ctab-file}
  \begin{syntax}
    ctab-file = <ctabset files>
  \end{syntax}
解析 \meta{ctabset files} 中的类别码表。这使得 \keyref*[high]{use-ctab} 可以使用这些类别码表。文件结构见\cref{sec:ctab}。
\end{keyval}

\begin{keyval}[path=high]{config-file,config-file+}
  \begin{syntax}
    config-file  = <config files>
    config-file+ = <config files>
  \end{syntax}
使用 \meta{config files} 设置的配置。若有重复的键,则只使用最后的那个。

文件类型的为 \href{https://toml.io/}{TOML},具体可用的配置见\cref{sec:config}。
\end{keyval}

\begin{keyval}[path=high]{gobble}
  \begin{syntax}
    gobble = <&auto|正整数> & 0
  \end{syntax}
每行最开始要忽略的字符数。如果为 \texttt{auto},则忽略的字符数为第一行开始的空格数。
\end{keyval}

\begin{keyval}[path=high]{tabs-len}
  \begin{syntax}
    tabs-len = <正整数> & 2
  \end{syntax}
要把水平制表符替换为多少空格。
\end{keyval}

\begin{keyval}[path=high]{lines}
  \begin{syntax}
    lines = <行号>       & 0
    lines = <开始, 结束>
  \end{syntax}
要高亮的代码行。为 \verb|0| 或 \verb|0,0| 时,高亮所有行,否则只高亮对应行(行号从 $1$ 开始),若为一个范围,则包括开始行,但不包括结束行。

\keyref*[high]{gobble} 为 \texttt{auto} 时,只会检测保留下来的行。
\end{keyval}

\begin{keyval}[path=high]{filename}
  \begin{syntax}
    filename = <文件> & \V\jobname.texhigh.verb
  \end{syntax}
要高亮的文件。设置它时,会把 \keyref*[high]{cache-dir} 附加到它前面。

使用 \env{texhigh} 环境时,会把代码保存到这个文件中。
\end{keyval}

\begin{keyval}[path=high]{cache-dir}
  \begin{syntax}
    cache-dir = <路径>
  \end{syntax}
设置缓存路径。如果是文件夹,则必须以 \texttt{/} 结尾,\pkg{texhigh} 不会自动附加 \texttt{/}。
且必须手动创建该文件夹,\pkg{texhigh} \emph{不会}创建它们,如果没有创建该文件夹,则会出错。
\end{keyval}

\begin{keyval}[path=high]{output}
  \begin{syntax}
    output = <文件>
  \end{syntax}
把 texhigh 的处理结果保存至 \meta{文件} 中。设置它时,会把 \keyref*[high]{cache-dir} 附加到它前面。
\end{keyval}

\begin{keyval}[path=high]{no-ctabs}
  \begin{syntax}
    no-ctabs
  \end{syntax}
清除已经设置的类别码表和 \keyref*[high]{use-ctab} 的值。
\end{keyval}

\begin{keyval}[path=high]{no-configs}
  \begin{syntax}
    no-configs
  \end{syntax}
清除已经设置的配置。
\end{keyval}

\begin{keyval}[path=high]{enhanced}
  \begin{syntax}
    enhanced = <&\TTF> & true
  \end{syntax}
是否启用增强模式,增强模式有更多功能。默认启用。
\end{keyval}

\begin{keyval}[path=high]{banner}
  \begin{syntax}
    banner = <&\TTF> & false
  \end{syntax}
是否输出 texhigh 的版权信息。一般无需修改。
\end{keyval}

\begin{keyval}[path=high]{text-base64}
  \begin{syntax}
    text-base64 = <&\TTF> & true
  \end{syntax}
是否把 \cs{texhightext} 高亮的代码以 base64 编码后再传入 texhigh。默认启用。一般无需修改。
\end{keyval}

\begin{keyval}[path=high]{kpse}
  \begin{syntax}
    kpse = <&\TTF> & false
  \end{syntax}
是否使用 \texttt{kpathsea} 查找 \keyref*[high]{config-file} 使用的文件。启用它后,texhigh 会\emph{显著地}变慢。建议使用 \texttt{kpsewhich} 找到对应的文件后,将其复制到工作目录,或使用绝对路径,而不是使用该选项。
\end{keyval}

\subsection{样式选项}

\begin{function}{\THSaveStyle}
  \begin{syntax}
    \V\THSaveStyle \marg{style} \marg{code}
  \end{syntax}
设置高亮风格。
\end{function}

\begin{function}{\THUseSavedStyle}
  \begin{syntax}
    \V\THUseSavedStyle \marg{style}
  \end{syntax}
使用由 \cs{THSaveStyle} 保存的风格。
\end{function}

\begin{keyval}[path=high]{style}
  \begin{syntax}
    style = <styles>
  \end{syntax}
使用由 \cs{THSaveStyle} 保存的风格。
\end{keyval}

\begin{keyval}[path=high]{line-number,linenos}
  \begin{syntax}
    line-number|linenos = <&\TTF> & false
  \end{syntax}
是否开启行号。
\end{keyval}

\begin{keyval}[path=high]{first-line-number,first-linenos}
  \begin{syntax}
    first-line-number|first-linenos = <起始行号>
  \end{syntax}
\pkg{texhigh} 每次高亮时不会重新设置行号,需要手动设置起始行号。
\end{keyval}

\begin{keyval}[path=high]{line-number-space,line-number-space+,line-number-space-}
  \begin{syntax}
    line-number-space  = <长度> & 3mm
    line-number-space+ = <长度>
    line-number-space- = <长度>
  \end{syntax}
设置(或增加,或减少)行号与代码之间的距离。
\end{keyval}

\begin{keyval}[path=high]{line-number-format}
  \begin{syntax}
    line-number-format = <code> & \V{\\scriptsize\\sffamily\ \#1}
  \end{syntax}
设置行号的格式。\verb|#1| 代表行号。

一般情况下,它不必使用 \tn{arabic} 等命令,而由 \keyref*[high]{the-line-number} 修改。
\end{keyval}

\begin{keyval}[path=high]{the-line-number}
  \begin{syntax}
    the-line-number = <code> & \V{\\arabic\{TeXHighLine\}}
  \end{syntax}
修改 \texttt{TeXHighLine} 计数器的显示方式。\texttt{TeXHighLine} 用于保存当前行号。

注意:\pkg{texhigh} 遇到新行时不会递增行号,只会在需要显示行号时才递增。
\end{keyval}

\begin{keyval}[path=high]{line-number-pos}
  \begin{syntax}
    line-number-pos = <&left|right|both> & both
  \end{syntax}
行号的位置。
\end{keyval}

\begin{keyval}[path=high]{line-kind}
  \begin{syntax}
    line-kind = <&enhanced|normal|none> & normal
  \end{syntax}
每个代码行的开始和结尾也可以自定义。默认情况下(\texttt{normal}),只会插入行号和换行。
如果设置为 \texttt{enhanced},则可以使用代码槽(socket)
\socket{texhigh/start-line}、\socket{texhigh/between-line}、\socket{texhigh/end-line} 来添加代码。
\end{keyval}

\begin{function}[module=socket point,type=socket point]{texhigh/start-line,texhigh/between-line,texhigh/end-line}
这三个代码槽(socket)会在指定位置执行。可通过 \cs{AssignSocketPlug} 来使用不同的插件(plug)。关于 socket 和 plug,见 \file{ltsockets-doc.pdf}。
\end{function}

\begin{function}{\texhigh@start@line}
钩子,它后面紧跟当前代码行。可在 \socket{texhigh/start-line} 代码槽中修改该命令用以检测当前代码行。
\end{function}

\begin{xample}
\makeatletter
\ExplSyntaxOn
\clist_const:Ne \c__my_comment_range_clist { \tl_to_str:n { comment, } }
\cs_new:Npn \@detect@start@comment #1
  {
    \tl_if_eq:nnTF {#1} { \THrs }
      { \@detect@start@comment@ }
      { \texhigh@default@start@line #1 }
  }
\cs_new:Npn \@detect@start@comment@ #1
  {
    \clist_if_in:NoF \c__my_comment_range_clist
      { \tl_to_str:n {#1} }
      { \texhigh@default@start@line }
    \THrs {#1}
  }
\ExplSyntaxOff
\NewSocketPlug{texhigh/start-line}{detect-start-comment}
  {\let\texhigh@start@line\@detect@start@comment}
\THSaveStyle{line-number-skip-comment}{%
  \SetKeys[texhigh/high]{line-kind=enhanced, linenos, first-linenos=1}%
  \AssignSocketPlug{texhigh/start-line}{detect-start-comment}%
}
\makeatother

\begin{texhigh}[style=line-number-skip-comment, gobble=auto]
  Non-Comment Line
  % A Comment Line
  % B Comment Line
  Non-Comment Line
    % C Comment Line but spaces
\end{texhigh}
\stopxamplecode
\SetKeys[texhigh/high]{lexer-catcode={explon,atletter}}
\xampleprint
\end{xample}

\begin{keyval}[path=high]{left-space,left-space+,left-space-,
  right-space,right-space+,right-space-}
  \begin{syntax}
    left-space  = <长度>
    left-space+ = <长度>
    left-space- = <长度>
  \end{syntax}
设置(或增加,或减少)代码距离文字左右两端的距离。
\end{keyval}

\begin{keyval}[path=high]{font,font+}
  \begin{syntax}
    font = <code> & \V{\ttfamily\raggedright}
  \end{syntax}
设置代码的字体。\pkg{texhigh} 还设置了 \verb|\linespread{1}|,可以在此选项内修改间距因子。
\end{keyval}

\begin{keyval}[path=high]{before-text,before-text+,extra-code,extra-code+}
  \begin{syntax}
    before-text = <code>
    extra-code  = <code>
  \end{syntax}
在设置完所有选项后执行 \meta{code}。
\end{keyval}

\begin{keyval}[path=high]{before,before+,after,after+}
  \begin{syntax}
    before = <code>
  \end{syntax}
\end{keyval}

钩子的执行顺序为 \meta{options} \meta{font} \meta{before-text} \meta{extra-code} \meta{before} \veta{highlighted code} \meta{after}。

\begin{keyval}[path=high]{save-catcode}
  \begin{syntax}
    save-catcode
    save-catcode = <code> &~
  \end{syntax}
用于保存当前的类别码。使用 \tn{@texhigh@reset@ctab} 可恢复到此处的类别码。\meta{code} 可以修改这个宏。默认情况下只会保存 \tn{dospecials} 里的字符的类别码。
\end{keyval}

\begin{function}{\texhighdefstyle}
  \begin{syntax}
    \V\texhighdefstyle   \marg{full path} \marg{options}
    \V\texhighdefstyle   \oarg{module} \marg{path} \marg{options}
    \V\texhighdefstyle * \marg{full path} \marg{code}
    \V\texhighdefstyle * \oarg{module} \marg{path} \marg{code}
  \end{syntax}
自定义额外的选项。不带星号的,\meta{options} 为同一 \meta{module} 下的选项,带星号的 \meta{code} 为代码。\meta{options} 和 \meta{code} 都可使用一个参数,为该选项的值。

\meta{module} 就是选项的灰色部分(不含末尾的 \texttt{/})。
例如,\keyref[high]{save-catcode} 的 \veta{module} 就是 \texttt{high},
\veta{path} 就是 \texttt{save-catcode},\veta{full path} 就是 \texttt{high/save-catcode}。
\end{function}

例如,使用
\begin{thverbatim}
  \texhighdefstyle[high]{my option 1}{
    first-line-number=1, line-number,
    line-number-pos=left, line-number-space=5mm,
  }
  \texhighdefstyle*[high]{my len 1}{\def\mylen{#1}}
\end{thverbatim}
后,就可以在 \cs{texhighfile} 等命令或环境中使用 \texhighverb|\texhighfile[my option 1, my len 1=10]{..}| 了。

\subsection{配置选项}

配置选项基本都有对应的 texhigh 的命令行选项和配置文件中的键。
它们当中有许多都是 texhigh 命令行选项的封装。

\begin{keyval}[path=high]{break-at}
  \begin{syntax}
    break-at = <{字符列表}> & \V{\\\ }, \V{\\\^\^I}
    break-at = \oarg{salt} \marg{字符列表}
  \end{syntax}
设置在哪些字符\emph{后}插入可断点。默认为空格和水平制表符。

\meta{salt} 为唯一标识符,由字母、数字等组成。
如果不给出 \meta{salt} 或两个 \meta{salt} 相同,则后面的那个会覆盖之前的。
\end{keyval}

\begin{keyval}[path=high]{char-replacements}
  \begin{syntax}
    char-replacements = \V\{ \meta{char}, ... \V\}
    char-replacements = \V\{ \meta{char}=\meta{repr}, ... \V\}
  \end{syntax}
高亮时替换 \meta{char},包括普通字符和控制序列名称\footnote{所谓控制序列名称就是除开转义符的其余部分,如 \tn{relax} 的名称就是 \texttt{relax}。}里的字符。

可以用 \cs{THSetCharReplacement} 来修改 \meta{char} 要被替换为什么。
例如,设置
\begin{thverbatim}
  char-replacements={\ =\textvisiblespace}
\end{thverbatim}
可以把空格替换为 \texttt{\textvisiblespace}(\tn{textvisiblespace})。

如果设置了 \meta{repr},相当于隐式使用 \cs{THSetCharReplacement}。
\end{keyval}

\begin{function}{\THSetCharReplacement}
  \begin{syntax}
    \V\THSetCharReplacement   \marg{char} \marg{repr}
    \V\THSetCharReplacement * \marg{char} \marg{repr}
  \end{syntax}
设置 \meta{char} 的替换文本,在显示 \meta{char} 是,会替换为 \meta{repr}。
若使用\emph{不}带星号的版本,则还会使用 \tn{ifincsname} 判断是否在控制序列的名里,若是,则不替换。
\end{function}

\begin{keyval}[path=high]{this-cs}
  \begin{syntax}
    this-cs = <code>
  \end{syntax}
设置所有控制序列的样式。
\end{keyval}

\begin{xample}
\def\thiscsstyle#1{{\THcolor{blue}#1}}
\texhighverb[this-cs=\thiscsstyle]{\relax}

\makeatletter
\def\thiscsstyle@#1#2{{\THcolor{blue}#1\underline{#2}}}
\def\thiscsstyle#1{\thiscsstyle@#1} % #1 为 {<escape char>}{<cs name>}
\makeatother
\texhighverb[this-cs=\thiscsstyle]{\relax}
\stopxamplecode
\xampleprint
\end{xample}

\begin{keyval}[path=high]{escape-inside}
  \begin{syntax}
    escape-inside = <{命令}>
    escape-inside = \meta{char_1}\meta{char_2}
  \end{syntax}
把 \meta{命令} 的参数“逃逸”(escape),或让 \meta{char_1} 和 \meta{char_2} 中间的内容“逃逸”。

所谓“逃逸”,就是指让它们不高亮,而保留其原始形式。
\end{keyval}

\begin{xample}
\begin{texhigh}[escape-inside=\E, escape-inside=&&]
  The \E{\textbf{text}} will be &\itshape escaped&.
  Even the \E{\verb|\relax|} works.
\end{texhigh}
\stopxamplecode
\xampleprint
\end{xample}

\begin{keyval}[path=high]{math-escape, comments-math-escape, non-comments-math-escape}
  \begin{syntax}
    math-escape
    math-escape = <&(any)|in-comments|non-comments>
    comments-math-escape &&
    non-comments-math-escape &&
  \end{syntax}
\texttt{math-escape} 把 \verb|$ $| 和 \verb|\( \)| 中间的内容作为数学公式。
值为 \texttt{in-comments} 时,它们需出现注释中才有效,
为 \texttt{non-comments},则不是出现在注释中才有效,
为 \texttt{any} 则不做限制,这也是默认情形。
\end{keyval}

\begin{xample}
\texhighverb[math-escape]{A \textit{cr} = $\rho \times \pi$ = \(\frac{N}{m}\)}.

\begin{texhigh}[gobble=auto, comments-math-escape]
  A \textit{cr} = $\rho \times \pi$ = \(\frac{N}{m}\)
  % A \textit{cr} = $\rho \times \pi$ = \(\frac{N}{m}\)
\end{texhigh}
\stopxamplecode
\xampleprint
\end{xample}

\begin{keyval}[path=high]{texcomments,texcl}
  \begin{syntax}
    texcomments|texcl = <&\TTF> & false
  \end{syntax}
使整个注释像正常文本那样不高亮,直接输出。
\end{keyval}

\begin{xample}
\begin{texhigh}[gobble=auto,texcomments]
  A \textbf{normal} text.
  % A \textbf{comment} text.
\end{texhigh}
\stopxamplecode
\xampleprint
\end{xample}

\begin{keyval}[path=high]{char-category*}
  \begin{syntax}
    char-category* = <{name}> <{regex}> <{style}>
  \end{syntax}
设置字符的类别及其样式。只对 token 为字符的才生效,不会影响控制序列名称里的字符。

\meta{name} 为名字,\meta{regex} 为(纯文本)正则表达式(见\cref{sec:regex}),\meta{style} 可使用 1 个参数,为当前字符。

当 \meta{style} 为 \cs{THPASS} 这个特殊值时,只设置类别而不修改样式,如果你已经定义了样式,而不想修改它,可以使用这个特殊值。
\end{keyval}

\begin{xample}
\texhighverb[char-category*={symbol}{[\ \$\&]}{\mbox{\THcolor{red}#1}}]
  |\def\a #1{$#1$} \a {\ b&\&}|
\stopxamplecode
\xampleprint
\end{xample}

\begin{xample}
\THSetCharReplacement{\ }{\textvisiblespace} % texhigh 已设置
\THSetCharReplacement{\$}{\S} % 把 $ 替换为 \S
\texhighverb[char-replacements={\ \$}, % 设置哪些字符要替换
  char-category*={symbol}{[\ \$\&]}{\mbox{\color{red}#1}}, % 修改颜色
] |\def\a #1{$#1$} \a {\ b&\&}|
\stopxamplecode
\xampleprint
\end{xample}

\begin{keyval}[path=high]{char-category}
  \begin{syntax}
    char-category = <&options..>
  \end{syntax}
设置字符的类别。\meta{options} 是(多个)可选值。目前 \pkg{texhigh} 没有提供任何可选值。
\end{keyval}

\begin{xample}
\texhighdefstyle[high]{char-category/my symbol}
  % 注意这里要用 ##1,因为 #1 代表 “my symbol”
  {char-category*={symbol}{[\ \$\&]}{\mbox{\THcolor{red}##1}}}

\texhighverb[char-category=my symbol]{\def\a #1{$#1$} \a {\ b&\&}}
\stopxamplecode
\xampleprint
\end{xample}

\begin{function}{\THSetCH,\THLetCH}
  \begin{syntax}
    \V\THSetCH \marg{name} \marg{code with 1 arg}
    \V\THLetCH \marg{name_1} \marg{name_2}
  \end{syntax}
设置字符的类别为 \meta{name} 的样式。或把 \meta{name_2} 样式复制给 \meta{name_1}。

\meta{code} 可使用 1 个参数,为当前字符。

\keyref*[high]{char-category*} 内部就是使用该命令。\cs{THPASS} 在此处无效。
\end{function}

\begin{keyval}[path=high]{cs-category,cs-category*}
  \begin{syntax}
    cs-category  = <{name}> <{cs name list}> <{style}>
    cs-category* = <{name}> <{regex}> <{style}>
  \end{syntax}
设置控制序列的类别及其样式。

\meta{name} 为名字,\meta{cs name list} 为控制序列名称列表,\meta{regex} 为(纯文本)正则表达式(见\cref{sec:regex}),\meta{style} 可使用 2 个参数,分别为转义符和控制序列名。

当 \meta{style} 为 \cs{THPASS} 这个特殊值时,只设置类别而不修改样式,如果你已经定义了样式,而不想修改它,可以使用这个特殊值。
\end{keyval}

\begin{function}{\THSetCS,\THLetCS}
  \begin{syntax}
    \V\THSetCS \marg{name} \marg{code with 2 args}
    \V\THLetCS \marg{name_1} \marg{name_2}
  \end{syntax}
设置控制序列的类别为 \meta{name} 的样式。或把 \meta{name_2} 样式复制给 \meta{name_1}。

\meta{code} 可使用 2 个参数,分别为转义符和控制序列名。比如控制序列 \token{relax},
转义符为 “\texttt\textbackslash”,控制序列名为 “\texttt{relax}”。

\keyreflist*[high]{cs-category,cs-category*} 内部就是使用该命令。\cs{THPASS} 在此处无效。
\end{function}

\begin{keyval}[path=high]{lexer,lexer*}
  \begin{syntax}
    lexer* = \marg{lexer cat_{start}} \marg{lexer cat_{end}} \marg{ctabs} \marg{action} \marg{value_{action}}
    lexer* = \oarg{salt} \marg{lexer cat_{start}} \marg{lexer cat_{end}} \marg{ctabs} \marg{action} \marg{value_{action}}
    lexer  = <&options..>
  \end{syntax}
用于设置如何解析要高亮的代码。

传递给 texhigh 的并非是 {\TeX} tokens,而是纯文本。由于 \TeX 是上下文有关的,
解析得到的 tokens 会根据前文不同而不同。在静态解析时,无法自动应用宏执行过程时触发的各类效果,
如类别码发生改变等等都不会被 texhigh 检测到。
为此,texhigh 支持通过检测当前所处的上下文来半自动应用宏执行后产生的效果。

\meta{lexer cat} 为触发 lexer 改变的条件。可以是纯文本、正则表达式、\TeX 正则表达式、位置(行、列号)等。正则表达式和 \TeX 正则表达式检测的内容为当前位置之前的所有符号。

每种 \meta{action} 对应 \meta{value} 的形式也不同。可用的 \meta{action} 为:
\begin{description}
  \item[\texttt{CatCode}] 修改类别码。\meta{value} 为类别码表名或 \meta{char}\texttt{=}\meta{catcode} 构成的列表。
  \item[\texttt{EndLine}] 修改行结束时插入的字符。\meta{value} 为数字。
\end{description}
\end{keyval}

\begin{keyval}[path=high]{lexer-catcode,lexer-catcode*,extra-catcode*}
  \begin{syntax}
    lexer-catcode* = \marg{lexer cat_{start}} \marg{lexer cat_{end}} \marg{value}
    lexer-catcode  = <&options..>
    extra-catcode* = \marg{value}
  \end{syntax}
\keyref*[high]{lexer*} 的 \meta{action} 为 \texttt{CatCode} 的特例。

\key{extra-catcode} 是 \meta{lexer cat_{start}} 为 \texttt{0},\meta{lexer cat_{end}} 为空的特例,相当于整个代码段都使用 \meta{value}。
\end{keyval}

\begin{keyval}[path=high]{no-lexer}
  \begin{syntax}
    no-lexer &&
  \end{syntax}
清除当 \meta{salt} 为空(或未设置)时的 \keyref*[high]{lexer*} 所设置的 lexer。
\end{keyval}

\begin{keyval}[path=high]{enabled-ranges,enabled-ranges+,remove-enabled-ranges,remove-enabled-ranges+}
  \begin{syntax}
    enabled-ranges = \marg{range name list}
    remove-enabled-ranges = \marg{range name list}
  \end{syntax}
texhigh 通过 range 机制检测命令的参数。range 就是一个特殊的代码片段,
可为不同的代码片段设置不同的样式。设置 range 并不会一定会启用它,texhigh 只会
启用指定的 range,这样可用把需要的 range 先写在配置文件中,然后按需启用。
\end{keyval}

\begin{keyval}[path=high]{range}
  \begin{syntax}
    range = \marg{full name} \marg{range actions}
  \end{syntax}
创建并\emph{启用} \meta{full name}。

range 可以分为普通代码段和逃逸代码段。普通代码段会被高亮,可以为其单独设置样式,也支持检测参数。
逃逸代码段不会被高亮,而会直接执行。
同一个 \meta{full name} 只能使用其中的一种,要么是普通代码段,要么是逃逸代码段。
不过,\meta{full name} 支持一种特殊的语法,当它为
\texttt{+}\marg{salt}\meta{real name} 时,
texhigh 认为其名为 \meta{real name},同时 \keyref*[high]{enabled-ranges} 也可以只用
\meta{real name},texhigh 会启用所有 \meta{full name},
只要其 \meta{real name} 包含在 \keyref*[high]{enabled-ranges} 中。
\end{keyval}

\meta{range actions} 也为键值选项,其 \meta{module} 为 \texttt{high/range-config}:
\begin{keyval}[path=high/range-config]{escape}
  \begin{syntax}
    escape = <&\TTF> &! false
  \end{syntax}
是否为逃逸代码段。
\end{keyval}

\begin{keyval}[path=high/range-config]{start,start*}
  \begin{syntax}
    start  = \marg{range cat}  &!
    start* = \marg{range cat*} &!
  \end{syntax}
此 range 何时开始。\meta{range cat} 为纯文本或正则表达式或 \TeX 正则表达式。检测的内容为当前位置及其后的所有内容。
\end{keyval}

\begin{keyval}[path=high/range-config]{arguments}
  \begin{syntax}
    arguments = \marg{arguments} &!
  \end{syntax}
\meta{arguments} 为 \pkg{ltcmd}(\pkg{xparse})的参数,参数数目最多为 9 个。

\meta{argument} 包括 \texttt{m o O d D r R s t v} 以及 \texttt{l u g G}
暂不支持 \texttt{e E c b},并且 \texttt{! + = >} 也是无效的。
此外,还支持一个特殊的符号 \verb|^^J|,它用于捕获当前行剩下的内容,带一个参数,
表示是否要求大括号成对存在。
\end{keyval}

\begin{keyval}[path=high/range-config]{remove-start}
  \begin{syntax}
    remove-start = <&\TTF> &! false
  \end{syntax}
移除该代码段的 \meta{start} 部分。仅为逃逸代码段时有效。
\end{keyval}

\begin{keyval}[path=high/range-config]{insert-brace}
  \begin{syntax}
    insert-brace = <&\TTF> &! false
  \end{syntax}
把该代码段的参数用一对 \texttt{\{ \}} 括起来。仅为逃逸代码段时有效。
\end{keyval}

\begin{keyval}[path=high/range-config]{use-argument}
  \begin{syntax}
    use-argument = <&\TTF> &! false
  \end{syntax}
移除参数最外层的括号或定界符。仅为逃逸代码段时有效。
\end{keyval}

\begin{keyval}[path=high/range-config]{insert-ending}
  \begin{syntax}
    insert-ending = <&\TTF> &! false
  \end{syntax}
若最后那个参数类型为 \texttt{u}、\texttt{U} 或 \texttt{\Verbatimize{\^\^J}} 时,
把标记参数结尾的符号移动到该代码段的后面。
若为逃逸代码段,则还需 \keyref*[high/range-config]{use-argument} 为 \texttt{true}。

例如,当使用参数 \texttt{\Verbatimize{\^\^J}} 时,标记该参数结尾的符号为换行符,
设置该选项为真时,会把该换行符移动到该代码段的后面,这样 texhigh 才能正确检测行的结束。
\end{keyval}

\begin{keyval}[path=high/range-config]{in-comments}
  \begin{syntax}
    in-comments = <&required|never|any> &! any
  \end{syntax}
是否仅在注释内才检测该 range。

\def\,{\hskip.16667em\relax}
值为 \texttt{required\,/\,must\,/\,true} 要求必须在注释内,
值为 \texttt{forbidden\,/\,never\,/\,prohibited\,/\,false} 要求不能在注释内,
值为 \texttt{irrelevant\,/\,any\,/\,dontcare} 则不做限制。
\end{keyval}

\begin{keyval}[path=high/range-config]{start-is-arg}
  \begin{syntax}
    start-is-arg = <&\TTF> &! false
  \end{syntax}
检测参数时是否包括 \meta{start}。
\end{keyval}

\begin{keyval}[path=high/range-config]{args-numbered}
  \begin{syntax}
    args-numbered = <&\TTF> &! false
  \end{syntax}
texhigh 把每个参数都作为单独的一个 range,名为 \texttt{argument.}\meta{arg type} 或
\texttt{argument.}\meta{arg pos},默认为前者,若
\keyref*[high/range-config]{args-numbered} 为真时,则使用后者。

\meta{arg type} 为参数类型,如 \texttt{m}、\texttt{D} 等,
\meta{arg pos} 为参数的位置,如 1、2、9 等,从 1 开始,最多为 9。
\end{keyval}

\begin{keyval}[path=high/range-config]{skip-if-pre,skip-if-post}
  \begin{syntax}
    skip-if-pre  = <{regex}> &!
    skip-if-post = <{regex}> &!
  \end{syntax}
在寻找完该 range 的参数后,若该 range 之前(或之后)的内容匹配 \meta{regex},则跳过该 range,继续寻找可能的 range。
\end{keyval}

\begin{function}{\THSetRange}
  \begin{syntax}
    \V\THSetRange \marg{range name}   \marg{start code}\oarg{end code}
    \V\THSetRange \marg{range name}   \oarg{range action} \marg{start code}\oarg{end code}
    \V\THSetRange \marg{range name} * \marg{start code}\oarg{end code}
    \V\THSetRange \marg{range name} * \oarg{range action} \marg{start code}\oarg{end code}
  \end{syntax}
设置 range 及其样式。

若不给出 \meta{range action},则只修改样式,否则还会创建并启用该 range。

\emph{不}带星号的版本会在 \meta{start code} 之前加上 \tn{begingroup} 以及 \tn{@texhigh@reset@ctab},在 \meta{end code} 之后加上 \tn{endgroup}。
\end{function}

\begin{function}{\THSetRS,\THSetRE,\THSetES,\THSetEE,\THLetRS,\THLetRE,\THLetES,\THLetEE}
  \begin{syntax}
    \V\THSetRS \marg{range name} \marg{start code}
    \V\THSetRE \marg{range name} \marg{end code}
  \end{syntax}
设置普通代码段和逃逸代码段的样式。代码段像环境一样,可以分别设置开始和结尾要执行的代码。

\cs{THSetRange} 就是同时设置普通代码段和逃逸代码段的样式。
\end{function}

\begin{function}{\THCollectRange}
  \begin{syntax}
    \V\THCollectRange \marg{do code} \V\THrs \meta{text code} \V\THre
    \V\THCollectRange \marg{do code} \V\THes \meta{text code} \V\THee
  \end{syntax}
收集普通代码段或逃逸代码段。\meta{do code} 后面紧跟着 \marg{text code}。
\end{function}

\begin{function}{\THPASS}
特殊的标记。
\end{function}

\begin{keyval}[path=high]{__config, __config*}
  \begin{syntax}
    __config  = \marg{raw key} \marg{raw value}
    __config* = \marg{raw key} \marg{raw value}
  \end{syntax}
设置配置。尽量不要使用它们。
\end{keyval}

\subsection{类别码表}\label{sec:ctab}

类别码表由方括号括起来的表名和 \meta{chars}\texttt{=}\meta{catcode} 组成。

预定义的类别码表位于 texhigh-rs 的\href{https://github.com/Sophanatprime/texhigh-rs}{仓库}的 \href{https://github.com/Sophanatprime/texhigh-rs/blob/main/src/prelude-ctabset.thcs}{\texttt{prelude-ctabset.thcs}} 内。

\subsection{配置文件}\label{sec:config}

配置文件的格式为 \href{https://toml.io/}{TOML}。

texhigh-rs 的\href{https://github.com/Sophanatprime/texhigh-rs}{仓库}里 \href{https://github.com/Sophanatprime/texhigh-rs/blob/main/src/prelude-config.toml}{\texttt{prelude-config.toml}} 给出了一个完整的配置文件。

\subsection{token 的分类}

texhigh 把不同的 token 分成几类(\veta{class}),每个类型都可设置不同的类别(\veta{category}),其中可设置样式的类型有:
\begin{description}\raggedright
\item[\texttt{bp}] 可断点;主要有两个类别:\texttt{cs} 和 \texttt{char},\texttt{cs} 的断点在控制序列前插入,\texttt{char} 在字符后插入;
\item[\texttt{cs}] 控制序列;除了由 \keyref*[high]{cs-category} 等设置之外,还有
  \prolist[、]{latex3.primitive,latex3.function.internal,latex3.function.public,latex3.function.kernel,latex3.variable.internal,latex3.variable.public,latex3.variable.kernel,latex.programming,latex.internal,latex.document,primitive.knuthtex,primitive.etex,primitive.pdftex,primitive.xetex,primitive.luatex,primitive.uptex,primitive.widely,primitive.sometex,primitive.luametatex};
\item[\texttt{ch}] 字符;除了由 \keyref*[high]{char-category} 等设置之外,对于组开始符和组结束符,有类别:\texttt{group.\-\meta{group level}};对于其它类别码不为 11 及 12 的字符,有类别:\texttt{catcode.\meta{catcode}};
\item[\texttt{rs}] 普通代码段的开始;除了由 \keyref*[high]{range} 设置的之外,还有 \prolist[、]{math.inline,comment};
\item[\texttt{re}] 普通代码段的结束;同 \texttt{rs};
\item[\texttt{st}] 文本;它仅包含类别码为 10、11、12 的字符;
\item[\texttt{es}] 逃逸代码段的开始;
\item[\texttt{ee}] 逃逸代码段的结束;
\item[\texttt{pn}] 标点。
\end{description}
它们都有对应的 \cs{THSet\meta{class}} 和 \cs{THLet\meta{class}} 命令用于设置样式,
其中 \meta{class} 需大写,如 \cs{THSetCS}。

\begin{function}{\THSetBP,\THLetBP,\THSetST,\THLetST,\THSetPN,\THLetPN}
  \begin{syntax}
    \V\THSetBP \marg{name} \marg{code}
    \V\THSetST \marg{name} \marg{code with 1 arg}
    \V\THSetPN \marg{name} \marg{code with 1 arg}
  \end{syntax}
设置对应类型的样式。
\end{function}

\begin{function}{\THRemoveClass,\THRemoveClasses}
  \begin{syntax}
    \V\THRemoveClass \marg{class} \marg{name list}
    \V\THRemoveClasses \marg{class list}
  \end{syntax}
\cs{THRemoveClass} 删除类型为 \meta{class} 的名为 \meta{name} 的样式。
\cs{THRemoveClasses} 删除类型为 \meta{class} 的除“\texttt{?}”之外的所有样式。
\end{function}

\begin{function}{\THSetFallback}
  \begin{syntax}
    \V\THSetFallback \marg{class} \marg{name} \marg{fallback names}
  \end{syntax}
如果 \meta{class} 的没有设置名为 \meta{name} 的样式,则依次使用 \meta{fallback names} 中的样式,如果它们都未设置,则使用“\texttt{?}”样式。
\end{function}


\section{辅助命令}

本节的命令一般可在设置样式时使用。

\begin{function}{\THcolor,\THColorStatus}
  \begin{syntax}
    \V\THcolor \marg{颜色表达式}
    \V\THcolor \oarg{色彩模式} \marg{color spec}
    \V\THColorStatus \marg{status}
  \end{syntax}
设置文本和填充的颜色。
与 \pkg{xcolor} 的 \tn{color} 命令相比,\cs{THcolor} 可以通过设置 \meta{status} 来启用或取消设置该颜色,如果 \meta{status} 为 0,则不设置颜色,否则设置对应颜色。
\end{function}

\begin{function}[EXP]{\texhigh@fallback,\texhigh@fallback@}
  \begin{syntax}
    \V{\texhigh@fallback}  \marg{class} \marg{name}
    \V{\texhigh@fallback@} \marg{class} \marg{name}
  \end{syntax}
这两个命令展开为没有发现名为 name 的样式时,要使用的样式。

\cs{texhigh@fallback@} 已经查找了其 fallback 列表中的样式,\cs{texhigh@fallback} 则还没有查找 fallback 列表。可以修改它们以使用不同的 fallback 策略。它们\emph{必须完全可展}。
\end{function}

\begin{function}[EXP]{\texhigh@cat@if@exists,\texhigh@cat@fallback}
  \begin{syntax}
    \V{\texhigh@cat@if@exists} \marg{class} \marg{cat name} \marg{true} \marg{false}
    \V{\texhigh@cat@fallback} \marg{class} \marg{cat name} \marg{false}
  \end{syntax}
\cs{texhigh@cat@if@exists} 检测 \meta{class} 是否有名为 \meta{cat name} 的样式。
\cs{texhigh@cat\-@fallback} 查找 \meta{class} 的 \meta{cat name} 的 fallback 列表,
展开为样式存在时的样式名,若这些样式都不存在,则使用 \meta{false}。
\end{function}

\begin{function}[EXP]{\texhigh@find@dotparent,\texhigh@dot@split}
  \begin{syntax}
    \V{\texhigh@dot@split} \marg{name}
    \V{\texhigh@find@dotparent} \marg{name}
  \end{syntax}
\cs{texhigh@dot@split} 把 \meta{name} 分成两份,第一份为最后一个句点之前的内容,第二份为最后一个句点之后的内容。这两份都用 \cs{exp_not:n} 保护起来。
\cs{texhigh@find@dotparent} 则只保留第一份。
\end{function}

\begin{xample}
\makeatletter
\def\printstr#1{\detokenize\expandafter{\expanded{#1}}}
\printstr{\texhigh@dot@split {\a.b.c.\d}}\quad
\printstr{\texhigh@find@dotparent {\a.b.c.\d}}\quad
\makeatother
\stopxamplecode
\SetKeys[texhigh/high]{extra-catcode*=atletter}
\xampleprint
\end{xample}

\begin{xample}
\makeatletter\ExplSyntaxOn
% 让它依次检测该样式的“父样式”,即最后一个句点之前的内容
% 如对于 a.b.c.d,会依次检测 a.b.c,a.b 和 a,若都不存在则使用 ?
% 我们修改 \texhigh@fallback@ 而非 \texhigh@fallback,这样它会首先检测 fallback 列表
\cs_set:Npn \texhigh@fallback@ #1#2
  { \__my_texhigh_fallback_dot:ne {#1} { \texhigh@find@dotparent {#2} } }
\cs_new:Npn \__my_texhigh_fallback_dot:nn #1 #2
  {
    \tl_if_empty:nTF {#2} { ? }
      {
        \texhigh@cat@if@exists {#1} {#2} {#2}
          { \__my_texhigh_fallback_dot:ne {#1} { \texhigh@find@dotparent {#2} } }
      }
  }
\cs_generate_variant:Nn \__my_texhigh_fallback_dot:nn { ne }
\ExplSyntaxOff\makeatother

\THRemoveClass{cs}{primitive.knuthtex}
\THSetFallback{cs}{primitive.knuthtex}{unknown-foo}
\THSetCS{primitive}{\mbox{\THcolor{blue}\bfseries #1#2}}
\texhighverb|\def|
\stopxamplecode
\SetKeys[texhigh/high]{extra-catcode*=atletter,extra-catcode*=explon}
\xampleprint
\end{xample}

\begin{function}{\@texhigh@reset@ctab,\@texhigh@reset@font}
钩子,分别用于重设类别码和字体。
\end{function}

\begin{function}{\@texhigh@rescan@lines}
  \begin{syntax}
    \V{\@texhigh@rescan@lines} \marg{balanced tokens}
    \V{\@texhigh@rescan@lines} \meta{token} \meta{tokens} \meta{token}
  \end{syntax}
重新扫描 \meta{tokens},可使得 \tn{verb} 等命令生效。
和 \tn{verb} 一样,\meta{tokens} 不能作为命令的参数。
\end{function}

\begin{function}{\@texhigh@underline}
  \begin{syntax}
    \V{\@texhigh@underline} \marg{text}
  \end{syntax}
给不可断行的 \meta{text} 加上下划线,该下划线的长度比 \meta{text} 的略短。
\end{function}

\begin{function}{\@texhigh@shadetext}
  \begin{syntax}
    \V{\@texhigh@shadetext} \marg{tikz options} \marg{text}
  \end{syntax}
给 \meta{text} 加上阴影或底纹。\meta{tikz options} 一般包含创建阴影或类似的选项,如 \texttt{left color=red, right color=blue} 或 \texttt{fill stretch image=image.png} 等。

需自行加载 \pkg{tikz} 宏包,或启用 \pkg{texhigh} 的 \texttt{tikz} 宏包选项。
\end{function}

\begin{function}[EXP]{\texhigh@replicate}
  \begin{syntax}
    \V{\texhigh@replicate} \marg{num} \marg{code}
  \end{syntax}
重复 \meta{code} \meta{num} 次。
\end{function}

\begin{function}{\texhigh@pdfliteral}
  \begin{syntax}
    \V{\texhigh@pdfliteral} \marg{pdf literal}
  \end{syntax}
\meta{pdf literal} 会被完全展开。
\end{function}

\begin{function}{\THmB,\THmC,\THmD,\THmH,\THmN,\THmP,\THmR,\THmS,\THmT}
特殊的标记:
\raisebox{\baselineskip}{\ttfamily
\begin{tabular}[t]{*{9}{c}}\toprule
  \V\THmB&\V\THmC&\V\THmD&\V\THmH&\V\THmN&\V\THmP&\V\THmR&\V\THmS&\V\THmT\\
  \V\\&\V\^&\V\$&\V\#&\V{\^\^J}&\V\%&\V{\^\^M}&\meta{space}&\meta{tab}\\\bottomrule
\end{tabular}}
\end{function}

\section{正则表达式}\label{sec:regex}

这里的正则表达式指的是纯文本正则表达式(简写为 \veta{regex}),也就是只对纯文本生效,此外还有 \TeX 正则表达式(简写为 \veta{regtex}),只对 tokens 生效,见\cref{sec:regtex}。

纯文本正则表达式的语法见 \url{https://docs.rs/regex/latest/regex/#syntax}。


\section{\TeX 正则表达式}\label{sec:regtex}

\TeX 正则表达式是作用于 {\TeX} tokens 的正则表达式,\pkg{l3regex} 就是此类。
与普通的正则表达式相比,它多了检测 token 的转义序列,例如 \verb|\c{relax}| 检测 token
是否为 \token{relax},只有 \token{relax} 才匹配此转义序列,普通正则表达式就做不到这一点。

texhigh 支持的 \veta{regtex} 的语法和 \pkg{l3regex} 很相似,但暂不支持
\verb|\b| \verb|\B| \verb|\G| \verb|\u| 这几个转义序列,
以及 \verb|\c| 转义序列的否定形式(即暂不支持 \verb|[^\c{begin}\c{end}]| 这类用法)。


\section{高亮原理}

\pkg{texhigh} 高亮代码有两步,首先把原始的代码传递给 texhigh 命令行工具,
它解析 \TeX 代码,检测 token 的分类以及类别,输出为特殊的 \TeX 代码,除了逃逸代码段外,
只有有限的几个控制序列:{\makeatletter\prolist[、][\@firstofone]{\cs{THls},\cs{THle},\cs{THin},\cs{THcr},\cs{THbp},\cs{THcs},\cs{THch},\cs{THrs},\cs{THre},\cs{THst},\cs{THes},\cs{THee},\cs{THpn}}},一般情况下不应重定义它们。然后根据这几个控制序列的参数不同得到不同的样式。


\section{实验特性}

\subsection{计算文字布局}

使用本节所述的特性需自行加载 \pkg{fontspec} 宏包。

texhigh 有 \cs{kaomoji} 命令用于排布文字,文字的显示效果部分取决于系统拥有的字体和所设置的字体。
\begin{function}{\kaomoji}
  \begin{syntax}
    \V\kaomoji   \marg{options} \marg{plain text}
    \V\kaomoji * \marg{options} \marg{plain text} \marg{image command}
  \end{syntax}
不带星号的命令会由 texhigh 计算布局(包括文字位置、字体等)再使用 \TeX 排版出来,由 \TeX 输出文字。
而带星号的命令由 texhigh 排版 \meta{plain text},将其输出为图片,\TeX 只负责导入该图片。

\meta{image command} 是加载图片的命令,如 \tn{includegraphics}\marg{options} 等。
\end{function}

\begin{keyval}[path=layout]{system-fonts}
  \begin{syntax}
    system-fonts = <&\TTF> & true
  \end{syntax}
是否使用系统字体。
\end{keyval}

\begin{keyval}[path=layout]{fonts,fonts+}
  \begin{syntax}
    fonts = <{font list}>
  \end{syntax}
设置文字可用的字体。
\end{keyval}

\begin{keyval}[path=layout]{fontsize,fontsize*}
  \begin{syntax}
    fontsize  = <{dim}>
    fontsize* = <{fontsize command}>
  \end{syntax}
设置字体大小。
\end{keyval}

\begin{keyval}[path=layout]{lineheight}
  \begin{syntax}
    lineheight = <{dim}>
  \end{syntax}
设置行高。
\end{keyval}

\begin{keyval}[path=layout]{force}
  \begin{syntax}
    force = <&\TTF> & true
  \end{syntax}
使用图片模式时,当图片已经存在时是否重新生成。
\end{keyval}

\begin{keyval}[path=layout]{cache-dir}
  \begin{syntax}
    cache-dir = <{directory}>
  \end{syntax}
同 \keyref*[high]{cache-dir}。
\end{keyval}

\setuplayout{preset=balance}
\printindex[docchange]
\printindex[docusage]

\end{document}